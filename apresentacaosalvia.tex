\documentclass[t]{beamer}
\usepackage[portuges]{babel}
\usepackage[utf8]{inputenc}
\usepackage[T1]{fontenc}
\usepackage{amsmath}
\usepackage{amssymb}
\usepackage{amsfonts}
\usepackage{amsthm}
\usepackage{graphicx}
\usepackage{xcolor}
\usepackage{booktabs}
\usepackage[scaled]{helvet}
\renewcommand{\familydefault}{\sfdefault}

%%%
%%% Define cores
%%%
\definecolor{cinza}{HTML}{75818B}
\definecolor{verdeufsc}{HTML}{006633}

%%%
%%% Configurações do Beamer
%%%
\setbeamertemplate{navigation symbols}{}
\setbeamersize{text margin left=10mm,text margin right=5mm} 

\setbeamercolor{frametitle}{fg=cinza}
\setbeamerfont{frametitle}{series=\bfseries}
\setbeamerfont{frametitle}{size=\huge}
\addtobeamertemplate{frametitle}{\vspace*{2mm}}{\vspace*{5mm}}

%%%
%%% Paginação
%%%
\addtobeamertemplate{navigation symbols}{}{
\usebeamerfont{footline}
\usebeamercolor[fg]{footline}
}
\setbeamercolor{footline}{fg=cinza}
\setbeamerfont{footline}{series=\bfseries}
\setbeamerfont{footline}{size=\tiny}
\setbeamertemplate{footline}{
\usebeamerfont{page number in head}
\usebeamercolor[fg]{page number in head}
\hspace{5mm}
\insertframenumber/\inserttotalframenumber
\vspace{5mm}
}

\setbeamertemplate{itemize items}[ball]
\setbeamertemplate{caption}[numbered]

%%%
%%% Background
%%%
\usebackgroundtemplate
{
\includegraphics[width=\paperwidth,height=\paperheight]{fundo.png}
}

%%%
%%% Estrutura (TOC)
%%%
\AtBeginSection[]{\frame{\frametitle{Roteiro}\tableofcontents[current]}}

\setbeamerfont{section in toc}{series=\bfseries}
\setbeamercolor{section in toc}{fg=gray}
\setbeamerfont{section in toc shaded}{series=\mdseries}
\setbeamercolor{section in toc shaded}{fg=gray!01}
\setbeamercolor{subsection in toc}{fg=cinza}
\setbeamercolor{subsection in toc shaded}{fg=gray!60}

\mode<presentation>

\begin{document}
%%% CORREÇÃO DO ERRO: Desativa a função especial das aspas no babel
\shorthandoff{"} 

%%%
%%% SLIDE 1: CAPA
%%%
{
\usebackgroundtemplate{\includegraphics[width=\paperwidth]{capa.png}}
\begin{frame}[plain]
\vspace{18mm}
\begin{flushright}
\textcolor{cinza}{\textbf{\huge{
Salvia sclarea: \\
Fronteiras da Ciência Global
}}}
\end{flushright}

\vspace{-6mm}
\begin{flushright}
\textcolor{cinza}{\textbf{\scriptsize{
Christian \& Eloísa Cerezolli
}}}
\end{flushright}

\vspace{-7mm}
\begin{flushright}
\textcolor{cinza}{\scriptsize{
Olericultura II | Agronomia | UFSC
}}
\end{flushright}
\end{frame}
}

%%%
%%% SLIDE 2: A PROMESSA
%%%
\begin{frame}
\frametitle{A Promessa}
\Large
\textbf{Ao final desta apresentação, vocês entenderão:}

\vspace{0.5cm}
\begin{itemize}
    \item Como a \textbf{China} está ``hackeando'' a biossíntese.
    \item Como o \textbf{Japão} descobriu a defesa oculta da planta.
    \item O novo mapa genético da \textbf{Coreia do Sul}.
    \item A agronomia de precisão da \textbf{Europa} para o cultivo.
\end{itemize}
\end{frame}

\section{Biotecnologia e Genética (Ásia)}

%%%
%%% SLIDE 3: CHINA
%%%
\begin{frame}
\frametitle{China: A Fábrica Celular}
\textit{Dalian Institute of Chemical Physics (CAS, 2025)}

\vspace{0.5cm}
\textbf{O Problema:}
\begin{itemize}
    \item Dependência da safra agrícola para o \textit{Ambrox} (perfumaria).
\end{itemize}

\vspace{0.3cm}
\textbf{A Solução (``The Hack''):}
\begin{itemize}
    \item Engenharia metabólica em leveduras (\textit{S. cerevisiae}).
    \item \textbf{Resultado:} Biossíntese de \textbf{Esclareol} em escala industrial, sem plantar um hectare.
\end{itemize}

\vspace{0.3cm}
\footnotesize{Fonte: ``Lifespan Engineering Strengthens Yeast Cell Factories...''}
\end{frame}

%%%
%%% SLIDE 4: JAPÃO
%%%
\begin{frame}
\frametitle{Japão: O Guarda-Costas}
\textit{NARO \& Universidades Japonesas}

\vspace{0.5cm}
\textbf{A Descoberta (Fujimoto et al.):}
\begin{itemize}
    \item O Esclareol não é apenas um aroma.
    \item Ele induz \textbf{resistência ativa} contra nematoides (\textit{Meloidogyne}).
\end{itemize}

\vspace{0.3cm}
\textbf{O Mecanismo:}
\begin{itemize}
    \item Aumento da lignificação das raízes (dependente de etileno).
    \item \textbf{Impacto:} Variedades com alto teor de óleo são mais resistentes a pragas de solo.
\end{itemize}
\end{frame}

%%%
%%% SLIDE 5: COREIA
%%%
\begin{frame}
\frametitle{Coreia: O Mapa do Tesouro}
\textit{Chungnam National University (2024/25)}

\vspace{0.5cm}
\textbf{O Marco Científico:}
\begin{itemize}
    \item Primeiro \textbf{Genoma Completo a Nível Cromossômico}.
\end{itemize}

\vspace{0.3cm}
\textbf{Por que importa?}
\begin{itemize}
    \item Permite Seleção Assistida por Marcadores (MAS).
    \item Controle preciso da razão \textit{Linalol} (Flor) vs. \textit{Acetato de Linalila} (Frescor).
\end{itemize}
\end{frame}

\section{Agronomia de Campo (Europa)}

%%%
%%% SLIDE 6: AGRONOMIA
%%%
\begin{frame}
\frametitle{Europa: Otimização no Campo}
\textit{Estudos da Itália (Sicília) e Ucrânia}

\begin{columns}[c]
\begin{column}{.5\textwidth}
\textbf{Itália (Qualidade):}
\begin{itemize}
    \item Estresse hídrico controlado favorece o acetato de linalila.
    \item Plasticidade fenotípica alta.
\end{itemize}
\end{column}

\begin{column}{.5\textwidth}
\textbf{Ucrânia (Produtividade):}
\begin{itemize}
    \item Ano 1: Estabelecimento.
    \item \textbf{Ano 2: Pico (15 ton/ha).}
    \item Ano 3: Queda brusca (10\% do pico).
    \item \textbf{Conclusão:} Rotação trienal obrigatória.
\end{itemize}
\end{column}
\end{columns}
\end{frame}

\section{Aplicação na Olericultura}

%%%
%%% SLIDE 7: TABELA TÉCNICA
%%%
\begin{frame}
\frametitle{Diretrizes de Cultivo (Padrão INRAE)}
\small
\begin{table}[]
\centering
\begin{tabular}{@{}ll@{}}
\toprule
\textbf{Parâmetro} & \textbf{Recomendação Científica} \\ \midrule
\textbf{Ciclo} & Bienal a perene curta (Foco no 2º ano). \\
\textbf{Solo} & Calcário, bem drenado. Tolera metais (cuidado!). \\
\textbf{Ponto de Colheita} & \textbf{Estágio de ``Leite'':} Sementes da base leitosas, \\
 & flores do topo abertas. \\
\textbf{Risco} & Colheita tardia reduz Linalol drasticamente. \\ \bottomrule
\end{tabular}
\end{table}

\vspace{0.5cm}
\textbf{Recomendação para Santa Catarina:}
\begin{itemize}
    \item Atenção à drenagem do solo (excesso de chuva prejudica o óleo).
\end{itemize}
\end{frame}

%%%
%%% SLIDE 8: CONCLUSÃO
%%%
\begin{frame}
\frametitle{Conclusão}
\Large
\textbf{O que levamos hoje:}

\vspace{0.5cm}
\begin{itemize}
    \item A \textit{Salvia sclarea} deixou de ser apenas uma cultura rústica.
    \item É um alvo de \textbf{alta tecnologia} genética e biossintética.
    \item O manejo agronômico deve focar na \textbf{colheita precisa} (estágio de leite) e rotação a cada 3 anos para viabilidade econômica.
\end{itemize}
\end{frame}

%%%
%%% SLIDE FINAL
%%%
{
\usebackgroundtemplate{\includegraphics[width=\paperwidth]{capa.png}}
\begin{frame}[plain]
\vspace{15mm}
\begin{center}
\textcolor{cinza}{\textbf{Obrigado}}
\end{center}

\vspace{-6mm}
\begin{center}
\textcolor{cinza}{\scriptsize{
Dúvidas?
}}
\end{center}

\vspace{10mm}
\begin{center}
\textcolor{cinza}{\scriptsize{
Christian (18201955) \& Eloísa Cerezolli \\
Agronomia UFSC
}}
\end{center}
\end{frame}
}

\end{document}