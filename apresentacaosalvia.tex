\documentclass[t]{beamer}
\usepackage[portuges]{babel}
\usepackage[utf8]{inputenc}
\usepackage[T1]{fontenc}
\usepackage{amsmath}
\usepackage{amssymb}
\usepackage{amsfonts}
\usepackage{amsthm}
\usepackage{graphicx}
\usepackage{xcolor}
\usepackage{booktabs}
\usepackage[scaled]{helvet}
\usepackage{verbatim}
\usepackage{amssymb}
\renewcommand{\familydefault}{\sfdefault}

%%%
%%% Define cores
%%%
\definecolor{cinza}{HTML}{75818B}
\definecolor{verdeufsc}{HTML}{006633}

%%%
%%% Configurações do Beamer
%%%
\setbeamertemplate{navigation symbols}{}
\setbeamersize{text margin left=10mm,text margin right=5mm} 

\setbeamercolor{frametitle}{fg=cinza}
\setbeamerfont{frametitle}{series=\bfseries}
\setbeamerfont{frametitle}{size=\huge}
\addtobeamertemplate{frametitle}{\vspace*{2mm}}{\vspace*{5mm}}

%%%
%%% Paginação
%%%
\addtobeamertemplate{navigation symbols}{}{
\usebeamerfont{footline}
\usebeamercolor[fg]{footline}
}
\setbeamercolor{footline}{fg=cinza}
\setbeamerfont{footline}{series=\bfseries}
\setbeamerfont{footline}{size=\tiny}
\setbeamertemplate{footline}{
\usebeamerfont{page number in head}
\usebeamercolor[fg]{page number in head}
\hspace{5mm}
\insertframenumber/\inserttotalframenumber
\vspace{5mm}
}

\setbeamertemplate{itemize items}[ball]
\setbeamertemplate{caption}[numbered]

%%%
%%% Background
%%%
\usebackgroundtemplate
{
\includegraphics[width=\paperwidth,height=\paperheight]{fundo.png}
}

%%%
%%% AQUI ESTAVA O COMANDO REMOVIDO (\AtBeginSection)
%%% Agora o roteiro não aparecerá automaticamente entre as seções.
%%%

\mode<presentation>

\begin{document}
% Desativa atalhos do babel para evitar erro de aspas
\shorthandoff{"} 

%%%
%%% SLIDE 1: CAPA
%%%
{
\usebackgroundtemplate{\includegraphics[width=\paperwidth]{capa.png}}
\begin{frame}[plain]
\vspace{15.5mm}
\begin{center}
\textcolor{cinza}{\textbf{\huge{
\textit{Salvia officinalis}
}}}
\end{center}

\vspace{-3mm}
\begin{center}
\textcolor{cinza}{\textbf{\scriptsize{
Christian e Eloísa Cerezolli
}}}
\end{center}

\vspace{-7mm}
\begin{center}
\textcolor{cinza}{\scriptsize{
Olericultura II | Agronomia | UFSC
}}
\end{center}
\end{frame}
}

%%%
%%% SLIDE 2: A PROMESSA
%%%
\begin{frame}
\frametitle{A Promessa}

\vspace{0.2cm}
\begin{itemize}
    \item A importância do nome científico.
    \item Como o Japão usa a Sálvia para regenerar pele humana.
    \item A ``assinatura metabólica'' descoberta pela China.
    \item O segredo do DNA Mitocondrial (revelado em 2023).
    \item A Regra das 16:00 que define o lucro na lavoura.
\end{itemize}
\end{frame}

\section{Fisiologia e Genética (Ásia)}
%%% slide 2: introdução salvia
%%%%
\begin{frame}
    \frametitle{Introdução}
\vspace{0.3cm}

    \begin{columns}[c] % O parâmetro [c] alinha verticalmente pelo centro
    
        % --- COLUNA DA ESQUERDA (TEXTO) ---
        \begin{column}{0.55\textwidth}
            \begin{itemize}
                \item \textit{Salvia officinalis L.} - O nome Salvia deriva do latim “salvare”.
                \vspace{0.3cm} % espaço entre itens
               \item Nativa do mediterrâneo - era usada pelos romanos para:\\
               \vspace{0.1cm} % espaço entre itens
                - Clarear e limpar os dentes com as folhas, para tirar o mau hálito, para escurecer os cabelos, para os nervos (tirar o tremor das mãos) e para excesso de suor.



               
   \end{itemize}
        \end{column}

        % --- COLUNA DA DIREITA (IMAGEM) ---
        \begin{column}{0.45\textwidth}
            \begin{center}
                % Ajustei a largura para preencher a coluna
                \includegraphics[width=1\textwidth]{mapamaior.png}
                \vspace{-5mm}
                
            \end{center}
        \end{column}
        
    \end{columns}
\end{frame}


\begin{frame}
%%%%caracteristicas
    \frametitle{Características}
    
    % Subtítulo ou contexto geral


    \begin{columns}[c] % O parâmetro [c] alinha verticalmente pelo centro
    
        % --- COLUNA DA ESQUERDA (TEXTO) ---
        \begin{column}{0.55\textwidth}
            \begin{itemize}
                \item Lamiaceae (alecrim, hortelã, manjericão);
                \vspace{0.3cm} % Espaço entre os itens
                \item Abelhas e mamangava
            \end{itemize}
        \end{column}

        % --- COLUNA DA DIREITA (IMAGEM) ---
        \begin{column}{0.45\textwidth}
            \begin{center}
                % Ajustei a largura para preencher a coluna
                \includegraphics[width=0.80\textwidth]{prancha-salvia.jpg}
                \vspace{3mm}
                
                % Link formatado (se tiver o pacote hyperref) ou apenas texto
                \tiny{Prancha botânica de \textit{Salvia officinalis}}
            \end{center}
        \end{column}
        
    \end{columns}
\end{frame}


%%%slide de todos os slides
\begin{frame}
\frametitle{Uso na atualidade}
    \begin{itemize}
        \item Japão: Biotecnologia Dérmica
        \item Cromossomos circulares
        \item Regra das 16h
        \item Tempero e chá
        \item Oléo essencial
    \end{itemize}
\end{frame}   


%%%
%%% SLIDE 3: JAPÃO
%%%

\begin{comment}
\begin{frame}
\frametitle{Japão: Biotecnologia Dérmica}
\textit{Journal of Cosmetics \& Dermatological Sciences (2022)}

\vspace{0.5cm}
\textbf{A Descoberta (Aioi et al.):}
\begin{itemize}
    \item Extratos de \textit{S. officinalis} ativam o canal iônico \textbf{TRPV4} em células humanas.
\end{itemize}

\vspace{0.3cm}
\textbf{O Mecanismo:}
\begin{itemize}
    \item A ativação regula o fluxo de Cálcio ($Ca^{2+}$).
    \item \textbf{Resultado:} Recuperação acelerada da barreira cutânea e formação de novas junções celulares.
    \item \textit{Não é apenas um chá, é um agente regenerativo.}
\end{itemize}
\end{frame}

%%%
%%% SLIDE 4: CHINA
%%%
\begin{frame}
\frametitle{China: A Assinatura Química}
\textit{Fudan University (2025)}

\vspace{0.5cm}
\textbf{O Estudo:}
\begin{itemize}
    \item Metabolômica Espacial comparando \textit{S. officinalis} vs. \textit{S. miltiorrhiza}.
\end{itemize}

\vspace{0.3cm}
\textbf{A Diferença Chave:}
\begin{itemize}
    \item Enquanto outras sálvias focam em anéis furanos...
    \item A \textit{S. officinalis} é uma fábrica de \textbf{Ácido Carnósico} e pequenos voláteis.
    \item \textbf{Aplicação:} Identificação molecular para evitar fraudes em extratos comerciais.
\end{itemize}
\end{frame}

%%%
%%% SLIDE 5: GENÉTICA
%%%
\begin{frame}
\frametitle{Genética: O Código Quebrado}
\textit{International Journal of Molecular Sciences (2023)}

\vspace{0.5cm}
\textbf{O Marco Científico:}
\begin{itemize}
    \item Sequenciamento completo do \textbf{Genoma Mitocondrial}.
\end{itemize}

\vspace{0.3cm}
\textbf{A Surpresa:}
\begin{itemize}
    \item O DNA não é um círculo único (como na maioria das plantas).
    \item Ele é dividido em \textbf{dois cromossomos circulares independentes}.
    \item Isso explica a alta plasticidade fenotípica e resistência da espécie.
\end{itemize}
\end{frame}

\section{Agronomia de Precisão}

%%%
%%% SLIDE 6: AGRONOMIA
%%%
\begin{frame}
\frametitle{Agronomia: A Regra das 16h}
\textit{Estudos de Variação Circadiana (Hazrati et al.)}

\begin{columns}[c]
\begin{column}{.5\textwidth}
\textbf{O Erro Comum:}
\begin{itemize}
    \item Colher pela manhã (padrão para folhosas).
    \item \textbf{Resultado:} Teor de óleo mínimo (0.59\%).
\end{itemize}
\end{column}

\begin{column}{.5\textwidth}
\textbf{A Ciência (Otimização):}
\begin{itemize}
    \item \textbf{Colheita Ideal: 16:00 -- 18:00.}
    \item Teor de óleo dobra para \textbf{1.14\%}.
    \item Ocorre o balanço ideal entre \textit{cis}-tujona e cânfora.
\end{itemize}
\end{column}
\end{columns}
\end{frame}
\end{comment}




%%% ciclo e desenvolvimento
\begin{frame}

 \frametitle{Ciclo e desenvolvimento}

\begin{columns}[c]
\begin{column}{.5\textwidth}
\begin{itemize}
        \item  \textbf{Germinação:} 7 a 21 dias, semeadura o ano todo.
    \item \textbf{Ciclo:} 90 dias no verão e 120 dias no inverno.
    \item \textbf{Transplante:} em dia nublado, quando as mudas tiverem 10 cm.
    \item \textbf{Espaçamento:} 80 cm entre linhas × 40 cm entre plantas.
    \item \textbf{Adaptação:} espécie perene e rústica, tolerante a diversas condições.
\end{itemize}
\end{column}

\begin{column}{.5\textwidth}
\begin{figure}
    \centering
    \includegraphics[width=0.8\linewidth]{fotosalvia.jpg}
    
    \label{fig:placeholder}
\end{figure}
\end{column}

\end{columns}
\end{frame}

\begin{frame}{Condições Ambientais}
\begin{columns}

% ----- COLUNA 1 -----
\begin{column}{0.5\textwidth}
\begin{itemize}
    \item \textbf{Solo:} bem drenado, rico em MO, evitar encharcamento.
    \item \textbf{Clima:} adapta-se a regiões quentes/temperadas; sensível à geada.
    \item \textbf{Luz:} mínimo de 6 h de sol direto/dia.
    \item \textbf{Rega:} moderada; regar apenas com solo seco ao toque.
\end{itemize}
\end{column}

% ----- COLUNA 2 -----
\begin{column}{0.5\textwidth}
\begin{itemize}
    \item \textbf{Adubação:} incorporar adubo orgânico no plantio; manter com MO.
    \item \textbf{Poda:} remover até 1/3 da planta; podas leves na primavera e pós-floração.
    \item \textbf{Sanidade:} retirar folhas secas para evitar míldio em alta umidade.
\end{itemize}
\end{column}

\end{columns}
\end{frame}


\section{Aplicação na Olericultura}
\begin{frame}
\frametitle{Propagação}
    \begin{center}
        % O comando \href cria o link na imagem
        \href{https://www.youtube.com/watch?v=QtpV7MqRGxk&t=275s}{
            % Coloque aqui um print do vídeo (salve como video_print.png)
            \includegraphics[width=0.8\textwidth]{video_print.png}
        }
        
    \end{center}
\end{frame}

%%%%culinaria

\begin{frame}{Culinária}

    
    \begin{columns}[c] % O [c] alinha verticalmente pelo centro
    
        % --- COLUNA DA ESQUERDA ---
        \begin{column}{0.5\textwidth}
            \centering
            % Substitua 'imagem1.png' pelo nome do seu arquivo
            \includegraphics[width=0.95\textwidth]{macarraodoido.jpg}
            
            \vspace{0.2cm} % Espaço entre imagem e texto
            \scriptsize{Legenda da imagem 1}
        \end{column}

        % --- COLUNA DA DIREITA ---
        \begin{column}{0.5\textwidth}
            \centering
            % Substitua 'imagem2.png' pelo nome do seu arquivo
            \includegraphics[width=0.5\textwidth]{nhoquezoom.jpg}
            
            \vspace{0.2cm}
            \scriptsize{Legenda da imagem 2}
        \end{column}

    \end{columns}
\end{frame}
    



%%% oleo essencial
\begin{frame}
    \frametitle{Oléo essencial}
    \begin{itemize}
        \item Tricomas glandulares $\rightarrow$ produção de óleo essencial
        \item Tricomas não glandulares $\rightarrow$ proteção física e aparência esbranquiçada
        \item Aparência esbranquiçada
    \end{itemize}
\end{frame}

%%%% para o ser humano

\begin{frame}
    \frametitle{Indicações para o ser humano}
     \hspace{-3mm}\includegraphics[width=1\textwidth]{indicacoessalvia.jpg}
        
\end{frame}

%%%
%%% SLIDE 7: mip
%%%
\begin{frame}
    \frametitle{Manejo Integrado de Pragas (MIP)}
    
    % Subtítulo ou contexto geral
    Óleo essencial de \textit{Salvia officinalis} no controle de caruncho do feijão
    \vspace{0.3cm}

    \begin{columns}[c] % O parâmetro [c] alinha verticalmente pelo centro
    
        % --- COLUNA DA ESQUERDA (TEXTO) ---
        \begin{column}{0.55\textwidth}
            \begin{itemize}
                \item Doses superiores a 0,5 L t$^{-1}$ de grãos aumentam as taxas de mortalidade para mais de 95\%, 6 horas após a aplicação.
                \vspace{0.3cm} % Espaço entre os itens
                \item O óleo essencial aplicado com 0,5 L t$^{-1}$ de grãos induz uma taxa de repelência acima de 90\%.
            \end{itemize}
        \end{column}

        % --- COLUNA DA DIREITA (IMAGEM) ---
        \begin{column}{0.45\textwidth}
            \begin{center}
                % Ajustei a largura para preencher a coluna
                \includegraphics[width=0.99\textwidth]{buracofeijao.jpg}
                \vspace{-5mm}
                
                % Link formatado (se tiver o pacote hyperref) ou apenas texto
                \tiny{\url{https://doi.org/10.1590/1983-40632016v4640034}}
            \end{center}
        \end{column}
        
    \end{columns}
\end{frame}






%%%
%%% SLIDE 8: CONCLUSÃO
%%%
\begin{frame}
\frametitle{Conclusão}
\vspace{0.5cm}
\begin{itemize}
    \item Importância do nome científico
    \item A \textit{Salvia officinalis} além de tempero
    \item algo sobre as pragas
    \item morfologia
\end{itemize}
\end{frame}

%%%
%%% SLIDE FINAL
%%%
{
\usebackgroundtemplate{\includegraphics[width=\paperwidth]{capa.png}}
\begin{frame}[plain]
\vspace{17mm}
\begin{center}
\textcolor{cinza}{\textbf{Christian e Eloísa Cerezolli \\
Agronomia UFSC}}
\end{center}


\end{frame}
}

\end{document}